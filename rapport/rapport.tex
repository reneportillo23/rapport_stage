\documentclass[12pt]{report}
\usepackage[utf8]{inputenc}
\usepackage[T1]{fontenc}
\usepackage[french]{babel}
\usepackage{geometry}
\geometry{a4paper, margin=2.5cm}
\usepackage{graphicx}
\usepackage{setspace}
\onehalfspacing

\begin{document}

% ------------------------ PAGE DE GARDE ------------------------
\begin{titlepage}
    \centering
    
    \textsc{\LARGE Université de Strasbourg}\\[0.5cm]
    \textsc{\Large Master Calcul Scientifique et Mathématiques de l'Innovation}\\[1.5cm]
    \textsc{\Large Année universitaire 2024-2025}\\[1.5cm]
    
    \rule{\linewidth}{0.5mm} \\[0.4cm]
    {\huge \bfseries Rapport de stage}\\[0.4cm]
    \rule{\linewidth}{0.5mm} \\[1.5cm]
    
    
    \begin{flushleft}
        \textbf{Nom :} José René PORTILLO\\
        \textbf{Formation :} Master 2 CSMI\\
        \textbf{Encadrant(e) académique :} Monsieur Christophe Prud'Homme\\
        \textbf{Tuteur de stage :} Moisés Rodriguez\\
        \textbf{Entreprise :} SHIFT Technology\\
        \textbf{Période :} Du 3 février 2025 au 1er août 2025
    \end{flushleft}
    
    \vfill
    
    
    
\end{titlepage}

% ------------------------ INTRODUCTION ------------------------

\chapter{Introduction}

Dans le cadre de ma dernière année au sein du Master de Mathématiques Appliquées, 
j'ai pu effectuer mon stage de fin d'études au sein de l'entreprise française Shift Technology, 
l'un des leaders mondiaux dans la détection de fraudes pour les assurances.

Ce rapport a pour objectif de décrire les travaux réalisés lors de ce stage, 
de présenter les enjeux de l'entreprise, les méthodes utilisées, ainsi que les résultats obtenus.

\section{Objectifs du stage}
\section{Présentation de l'entreprise}
\subsection{Le métier de Data Scientist chez Shift Technology}
\subsection{Outils et logiciels}

\section{La détéction de la fraude via les "rules based" scenario}
\subsection{Différents scénarios}
\subsection{Résultats}
\section{La détéction de la fraude via des modèles de Machine Learning}
\subsection{Light GBM et Random Forrest}
\subsection{Profilling des variables}
\subsection{Mise en production et résultats}
\section{Au-delà des Mathématiques}

\section{Conclusion}



\end{document}
